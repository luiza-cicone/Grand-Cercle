\documentclass[a4paper, 11px]{article}

\usepackage[french]{babel}
\usepackage[utf8]{inputenc}
\usepackage{fancyhdr}
\usepackage{lastpage}
\usepackage{graphicx}
\usepackage{rotating}
\usepackage{textcomp}
\usepackage{xspace}
\usepackage[toc,page]{appendix}
\usepackage{array}
\usepackage{amssymb}
\usepackage{enumerate}
\usepackage{enumitem}
\usepackage{eso-pic}

% \usepackage{needspace}


%%%%%%
 
\usepackage{listings}

\lstset{
  morekeywords={},
  sensitive=f,
  morecomment=[l]--,
  morestring=[d]",
  showstringspaces=false,
  basicstyle=\small\ttfamily,
  keywordstyle=\bf\small,
  commentstyle=\itshape,
  stringstyle=\sf,
  extendedchars=true,
  columns=[c]fixed
}

% CI-DESSOUS: conversion des caractères accentués UTF-8 
% en caractères TeX dans les listings...
\lstset{
  literate=%
  {À}{{\`A}}1 {Â}{{\^A}}1 {Ç}{{\c{C}}}1%
  {à}{{\`a}}1 {â}{{\^a}}1 {ç}{{\c{c}}}1%
  {É}{{\'E}}1 {È}{{\`E}}1 {Ê}{{\^E}}1 {Ë}{{\"E}}1% 
  {é}{{\'e}}1 {è}{{\`e}}1 {ê}{{\^e}}1 {ë}{{\"e}}1%
  {Ï}{{\"I}}1 {Î}{{\^I}}1 {Ô}{{\^O}}1%
  {ï}{{\"i}}1 {î}{{\^i}}1 {ô}{{\^o}}1%
  {Ù}{{\`U}}1 {Û}{{\^U}}1 {Ü}{{\"U}}1%
  {ù}{{\`u}}1 {û}{{\^u}}1 {ü}{{\"u}}1%
}

%%%%%%%%%%
% TAILLE DES PAGES (A4 serré)

\setlength{\parindent}{1cm}
\setlength{\parskip}{1ex}
\setlength{\textwidth}{17cm}
\setlength{\textheight}{22,7cm}
\setlength{\oddsidemargin}{-.7cm}
\setlength{\evensidemargin}{-.7cm}


\renewcommand{\labelenumi}{\arabic{enumi}.} 
\renewcommand{\labelenumii}{\arabic{enumi}.\arabic{enumii}}
\renewcommand{\labelenumiii}{\arabic{enumi}.\arabic{enumii}.\arabic{enumiii}}

%%%%%%%%%%

\newcommand\BackgroundPic{
\put(0,0){
\parbox[b][\paperheight]{\paperwidth}{%
\vfill
\includegraphics[width=\paperwidth,height=\paperheight,
keepaspectratio]{background.jpg}%
\vfill
}}}
%%%%%%%

\begin{document}

\AddToShipoutPicture{\BackgroundPic}


\renewcommand{\appendixtocname}{Annexes}
\DeclareGraphicsExtensions{.pdf,.png,.jpg}

\begin{titlepage}
\setlength{\parindent}{0cm}

\begin{center}

% Upper part of the page
 \begin{figure}[!h]
\includegraphics[bb=-550 -10 -250 20, scale=0.7]{./logo.pdf}
\end{figure}
% logo.pdf: 612x792 pixel, 72dpi, 21.59x27.94 cm, bb=0 0 612 792


\vspace{4cm}
\rule{\linewidth}{.5pt}
\vspace{2mm}


\begin{center}
{\LARGE GRAND CERCLE MOBILE - GCM}

\vspace{1cm}


{\Huge \bf CAHIER DES CHARGES}


\end{center}


\vspace{1cm}

%===================================================================
\begin{center}
$ $\\
\large{ \textbf{Luiza CICONE - Jérémy KREIN - Jérémy LUQUET - Paul MAYER}}\\
\large{ \textbf{ISI - IF}}
$ $\\
\end{center}
\rule{\linewidth}{.5pt}


\vfill

% Bottom of the page

{\large Mai 2012}

\end{center}
\end{titlepage}

\tableofcontents

\newpage


\section{Définition du besoin}

\subsection{Scénario pour mettre en avant le problème}

La vie associative à Grenoble INP est très riche, une multitude d'associations sont présentes pour dynamiser la vie étudiante.
Les événements proposés aux étudiants sont très nombreux, et les différents outils de communication les promouvant ne sont pas adaptés.
Actuellement les principaux outils utilisés pour la communication sont les mails, les affiches et facebook.
En ce qui concerne les mails, d'une part les étudiants ne les regardent pas toujours régulièrement et ne prennent pas le temps de les lire attentivement et
d'autre part, les informations concernant les événements dans ces mails sont souvent noyées au milieu d'autres informations (libération de colocation, etc.).
Pour ce qui est de facebook, les informations sont perdues au milieu de tous les commentaires, les événements en tout genre ayant lieu dans toute
la france, etc. Pour les affiches, nous ne pouvons pas en imprimer beaucoup et donc seulement les étudiants directement visés ont des affiches dans
leur batîment. Si un étudiant Phelma aimerait participer à un événement sportif de l'ense3, il ne peut s'en informer grâce l'affiche dans son école.

{\bf Scénario 1} 

Alexandre est un nouvel étudiant à l'Ense3 et il est très intéressé par les événements sportifs. Il pourra bien sûr profiter des événements organisés par
le Bureau Des Sports de son école qui sont promus dans les batîments de l'ense3. Cependant il n'aura pas directement d'informations sur les événements sportifs 
organisés par les autres écoles de Grenoble INP qui pourraient l'intéresser. Il ne recevera pas de mail à ce sujet et ne sera pas invité sur facebook
à l'événement car il n'aura pas encore les différents bureaux des élèves en amis sur le réseau. De plus, pendant la période d'intégration les étudiants n'ont vraiment pas beaucoup de temps pour eux (cours, événements, paperasse, etc.) et Alexandre ne peut se permettre de passer 3 heures sur facebook pour chercher des événements
sur les nombreuses pages des différentes associations de Grenoble INP. Grâce à notre implication, il pourra de n'importe ou avoir toutes les informations qu'il désire
à propos des événements sportifs (il choisira grâce à un filtre) pendant toute l'année. Une notification lui sera alors envoyée à chaque ajout d'événement. Il pourra donc
instantanément être au courant.

{\bf Scénario 2}

Des amis se retrouvent en ville pour passer la soirée ensemble. Après quelques verres dans un bar ils ne savent pas quoi faire par la suite. Un des membres du groupe
prend son smartphone. Il ne va pas passer sa soirée à chercher quelquechose à faire sur facebook ou dans ses nombreux mails. En revanche, en lançant notre application,
il peut consulter en quelques secondes la liste des événements se déroulant pendant la soirée et les informations à leur sujet. Il apprend ainsi qu'il y a un match d'improvisation dans un café tout près, ils peuvent alors s'y rendre et passer une bonne soirée.

\subsection{Modèle de l'utilisateur}

Nos utilisateurs seront des étudiants de Grenoble INP. Au sein de ce groupe, nous pouvons différencier plusieurs profils différents.

D'un côté nous avons les étudiants qui s'intéressent beaucoup à la vie associative de Grenoble INP et qui sont engagés dans des associations.
Ils sont souvent très intéressés par tous les événements que réalisent les différentes associations de Grenoble INP, que ce soit pour y participer ou seulement
pour en être informé. Ce profil d'utilisateur veut donc pouvoir être au courant de tout ce qui fait l'actualité.

De l'autre côté, nous avons des étudiants qui ont chacun leurs goûts et occupations propres. Ils sont intéressés uniquement par des événements de type précis (soirée, événement sportif) ou d'événements d'associations précises (leur BDE, etc.]. Il faudra donc que notre application leur offre la possibilité de choix sur les événements dont ils seront informés.


\newpage


\section{Vision de la solution}
Nous avons choisi de développer une application smartphone dédiée à la communication du Grand Cercle et de la vie associative de Grenoble INP.

\newpage

\section{Méthodes}

\subsection{Étude de la concurrence}
La vie associative à Grenoble INP est dense, et toutes les associations - ou presque - possèdent leur propre site internet ou ont un espace dédié sur le site u Grand Cercle.
Cependant, malgré une visibilité importante sur internet, aucune association n'a encore développé d'application pour smartphone. Cette application serait donc une première du point de vue de la communication pour un association à Grenoble INP.

\subsection{Étude qualitative}
Nous avons interviewé plusieurs étudiants de Grenoble INP avec des profils différents afin de mieux cerner les attentes des utilisateurs potentiels de cette application smartphone. Pour ne pas biaiser notre étude, nous avons préparé un questionnaire précis à réponses ouvertes.
Les entretiens ont tous été enregistrés afin de ne pas oublier une idée importante lors de la restitution des idées. 

\subsection{Étude quantitative}
En considérant les résultats obtenus lors de la restitution des idées obtenues lors de l'étude qualitative, nous avons élaboré un second questionnaire. Ce questionnaire a permis d'avoir des avis plus nombreux sur ces différentes idées, et ainsi d'évaluer les fonctionnalités qui paraissent indispensables.

\newpage

\section{Bilan}

\subsection{Exigences fonctionnelles}


\subsection{Exigences non fonctionnelles}


\appendix
\addappheadtotoc

\newpage

\section{Questionnaire utilisé pour l'étude qualitative}

bla bla..
\newpage

\section{Compte-rendus des entretiens}

bla bla..
\newpage

\section{Questionnaire utilisé pour l'étude quantitative}


\end{document}

