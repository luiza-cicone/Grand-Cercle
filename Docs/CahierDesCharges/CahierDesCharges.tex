\documentclass[a4paper, 11px]{article}

\usepackage[french]{babel}
\usepackage[utf8]{inputenc}
\usepackage{fancyhdr}
\usepackage{lastpage}
\usepackage{graphicx}
\usepackage{rotating}
\usepackage{textcomp}
\usepackage{xspace}
\usepackage[toc,page]{appendix}
\usepackage{array}
\usepackage{amssymb}
\usepackage{enumerate}
\usepackage{enumitem}
\usepackage{eso-pic}

% \usepackage{needspace}


%%%%%%
 
\usepackage{listings}

\lstset{
  morekeywords={},
  sensitive=f,
  morecomment=[l]--,
  morestring=[d]",
  showstringspaces=false,
  basicstyle=\small\ttfamily,
  keywordstyle=\bf\small,
  commentstyle=\itshape,
  stringstyle=\sf,
  extendedchars=true,
  columns=[c]fixed
}

% CI-DESSOUS: conversion des caractères accentués UTF-8 
% en caractères TeX dans les listings...
\lstset{
  literate=%
  {À}{{\`A}}1 {Â}{{\^A}}1 {Ç}{{\c{C}}}1%
  {à}{{\`a}}1 {â}{{\^a}}1 {ç}{{\c{c}}}1%
  {É}{{\'E}}1 {È}{{\`E}}1 {Ê}{{\^E}}1 {Ë}{{\"E}}1% 
  {é}{{\'e}}1 {è}{{\`e}}1 {ê}{{\^e}}1 {ë}{{\"e}}1%
  {Ï}{{\"I}}1 {Î}{{\^I}}1 {Ô}{{\^O}}1%
  {ï}{{\"i}}1 {î}{{\^i}}1 {ô}{{\^o}}1%
  {Ù}{{\`U}}1 {Û}{{\^U}}1 {Ü}{{\"U}}1%
  {ù}{{\`u}}1 {û}{{\^u}}1 {ü}{{\"u}}1%
}

%%%%%%%%%%
% TAILLE DES PAGES (A4 serré)

\setlength{\parindent}{1cm}
\setlength{\parskip}{1ex}
\setlength{\textwidth}{17cm}
\setlength{\textheight}{22,7cm}
\setlength{\oddsidemargin}{-.7cm}
\setlength{\evensidemargin}{-.7cm}


\renewcommand{\labelenumi}{\arabic{enumi}.} 
\renewcommand{\labelenumii}{\arabic{enumi}.\arabic{enumii}}
\renewcommand{\labelenumiii}{\arabic{enumi}.\arabic{enumii}.\arabic{enumiii}}

%%%%%%%%%%

\newcommand\BackgroundPic{
\put(0,0){
\parbox[b][\paperheight]{\paperwidth}{%
\vfill
\includegraphics[width=\paperwidth,height=\paperheight,
keepaspectratio]{background.jpg}%
\vfill
}}}
%%%%%%%

\begin{document}

\AddToShipoutPicture{\BackgroundPic}


\renewcommand{\appendixtocname}{Annexes}
\DeclareGraphicsExtensions{.pdf,.png,.jpg}

\begin{titlepage}
\setlength{\parindent}{0cm}

\begin{center}

% Upper part of the page
 \begin{figure}[!h]
\includegraphics[bb=-550 -10 -250 20, scale=0.7]{./logo.pdf}
\end{figure}
% logo.pdf: 612x792 pixel, 72dpi, 21.59x27.94 cm, bb=0 0 612 792


\vspace{4cm}
\rule{\linewidth}{.5pt}
\vspace{2mm}


\begin{center}
{\LARGE GRAND CERCLE MOBILE - GCM}

\vspace{1cm}


{\Huge \bf CAHIER DES CHARGES}


\end{center}


\vspace{1cm}

%===================================================
\begin{center}
$ $\\
\large{ \textbf{Luiza CICONE - Jérémy KREIN - Jérémy LUQUET - Paul MAYER}}\\
\large{ \textbf{ISI - IF}}
$ $\\
\end{center}
\rule{\linewidth}{.5pt}


\vfill

% Bottom of the page

{\large Mai 2012}

\end{center}
\end{titlepage}

\tableofcontents

\newpage


\section{Définition du besoin}

\subsection{Scénario pour mettre en avant le problème}

La vie associative à Grenoble INP est très riche, une multitude d'associations sont présentes pour dynamiser la vie étudiante.
Les événements proposés aux étudiants sont très nombreux, et les différents outils de communication les promouvant ne sont pas adaptés.

Actuellement les principaux outils utilisés pour la communication sont les mails, les affiches et Facebook.
En ce qui concerne les mails, d'une part les étudiants ne les regardent pas toujours régulièrement et ne prennent pas le temps de les lires attentivement et
d'autre part, les informations concernant les événements dans ces mails sont souvent noyées au milieu d'autres informations (libération de colocation, etc.).
Sur Facebook les informations sont perdues au milieu de tous les commentaires, les événements en tout genre ayant lieu dans toute
la France, etc.
Les affiches ont aussi leurs limitations, nous ne pouvons pas en imprimer beaucoup et donc seulement les étudiants directement visés ont des affiches dans leur bâtiment. 
Par exemple un étudiant Phelma ne pourra pas s'informer par rapport à un événement sportif de l'Ense3, comme il s'agit des écoles dans des bâtiments différents.


{\bf Scénario 1} 

Alexandre est un nouveau étudiant à l'Ense3 et il est très intéressé par les événements sportifs. Il a participé aux événements organisés par le Bureau de Sport de son école, mais il aimerai pouvoir participer aux événements sportifs d'autres écoles qui ne sont pas directement promu dans son école. Comme il est en première année, il passe beaucoup de temps à l'école, il s'occupe des taches administratives (inscription, assurance, etc.) et donc il n'a pas accès facilement à son ordinateur pour chercher les événements qui l'intéressent. 

\textit{Il y a des étudiants ont besoin d'être informés facilement sur certains types des événements}


{\bf Scénario 2}

Des amis se retrouvent en ville pour passer la soirée ensemble. Après passer un peu de temps dans un bar ils cherchent quelque chose à faire pour la suite. Marion sort sont iPhone pour chercher sur Facebook les événements de ce soir pour pouvoir choisir ensemble. Elle se perd dans tous les informations inutiles partagés par ses amis et aussi elle a pas accès a certaines informations comme elle n'est pas connecté avec certains associations.

\textit{Certains étudiants ont besoin d'avoir accès partout aux informations concernant les associations de Grenoble INP}


\subsection{Modèle de l'utilisateur}

Nos utilisateurs seront des étudiants de Grenoble INP. Au sein de ce groupe, nous pouvons différencier plusieurs profils différents.

D'un côté nous avons les étudiants qui s'intéressent beaucoup à la vie associative de Grenoble INP et qui sont engagés dans des associations. Ils sont souvent très intéressés par tous les événements que réalisent les différentes associations de Grenoble INP, que ce soit pour y participer ou seulement
pour en être informé. Ce profil d'utilisateur veut donc pouvoir être au courant de tout ce qui fait l'actualité.


De l'autre côté, nous avons des étudiants qui ont chacun leurs goûts et occupations propres. Ils sont intéressés uniquement par des événements d'un certain type (soirée, événement sportif; etc) ou d'événements d'associations précises (leur BDE, le club Rock]. Alors eux il veulent être informés les événements de leur choix.

\newpage


\section{Vision de la solution}
Nous avons choisi de développer une application smartphone dédiée à la communication du Grand Cercle et de la vie associative de Grenoble INP. Les objectives principaux..

\newpage

\section{Méthodes}

\subsection{Étude de la concurrence}
La vie associative à Grenoble INP est dense, et toutes les associations - ou presque - possèdent leur propre site internet ou ont un espace dédié sur le site u Grand Cercle.
Cependant, malgré une visibilité importante sur internet, aucune association n'a encore développé d'application pour smartphone. Cette application serait donc une première du point de vue de la communication pour un association à Grenoble INP.

\subsection{Étude qualitative}
Nous avons interviewé plusieurs étudiants de Grenoble INP avec des profils différents afin de mieux cerner les attentes des utilisateurs potentiels de cette application smartphone. Pour ne pas biaiser notre étude, nous avons préparé un questionnaire précis à réponses ouvertes.
Les entretiens ont tous été enregistrés afin de ne pas oublier une idée importante lors de la restitution des idées. 

\subsection{Étude quantitative}
En considérant les résultats obtenus lors de la restitution des idées obtenues lors de l'étude qualitative, nous avons élaboré un second questionnaire. Ce questionnaire a permis d'avoir des avis plus nombreux sur ces différentes idées, et ainsi d'évaluer les fonctionnalités qui paraissent indispensables.

\newpage

\section{Bilan}

\subsection{Exigences fonctionnelles}


\subsection{Exigences non fonctionnelles}


\appendix
\addappheadtotoc

\newpage

\section{Questionnaire utilisé pour l'étude qualitative}

bla bla..
\newpage

\section{Compte-rendus des entretiens}

bla bla..
\newpage

\section{Questionnaire utilisé pour l'étude quantitative}


\end{document}

