\documentclass[a4paper, 11px]{article}

\usepackage[french]{babel}
\usepackage[utf8]{inputenc}
\usepackage{fancyhdr}
\usepackage{lastpage}
\usepackage{graphicx}
\usepackage{rotating}
\usepackage{textcomp}
\usepackage{xspace}
\usepackage[toc,page]{appendix}
\usepackage{array}
\usepackage{amssymb}
\usepackage{enumerate}
\usepackage{enumitem}
\usepackage{eso-pic}

% \usepackage{needspace}


%%%%%%
 
\usepackage{listings}

\lstset{
  morekeywords={},
  sensitive=f,
  morecomment=[l]--,
  morestring=[d]",
  showstringspaces=false,
  basicstyle=\small\ttfamily,
  keywordstyle=\bf\small,
  commentstyle=\itshape,
  stringstyle=\sf,
  extendedchars=true,
  columns=[c]fixed
}

% CI-DESSOUS: conversion des caractères accentués UTF-8 
% en caractères TeX dans les listings...
\lstset{
  literate=%
  {À}{{\`A}}1 {Â}{{\^A}}1 {Ç}{{\c{C}}}1%
  {à}{{\`a}}1 {â}{{\^a}}1 {ç}{{\c{c}}}1%
  {É}{{\'E}}1 {È}{{\`E}}1 {Ê}{{\^E}}1 {Ë}{{\"E}}1% 
  {é}{{\'e}}1 {è}{{\`e}}1 {ê}{{\^e}}1 {ë}{{\"e}}1%
  {Ï}{{\"I}}1 {Î}{{\^I}}1 {Ô}{{\^O}}1%
  {ï}{{\"i}}1 {î}{{\^i}}1 {ô}{{\^o}}1%
  {Ù}{{\`U}}1 {Û}{{\^U}}1 {Ü}{{\"U}}1%
  {ù}{{\`u}}1 {û}{{\^u}}1 {ü}{{\"u}}1%
}

%%%%%%%%%%
% TAILLE DES PAGES (A4 serré)

\setlength{\parindent}{1cm}
\setlength{\parskip}{1ex}
\setlength{\textwidth}{17cm}
\setlength{\textheight}{22,7cm}
\setlength{\oddsidemargin}{-.7cm}
\setlength{\evensidemargin}{-.7cm}


\renewcommand{\labelenumi}{\arabic{enumi}.} 
\renewcommand{\labelenumii}{\arabic{enumi}.\arabic{enumii}}
\renewcommand{\labelenumiii}{\arabic{enumi}.\arabic{enumii}.\arabic{enumiii}}

%%%%%%%%%%

\newcommand\BackgroundPic{
\put(0,0){
\parbox[b][\paperheight]{\paperwidth}{%
\vfill
\includegraphics[width=\paperwidth,height=\paperheight,
keepaspectratio]{background.jpg}%
\vfill
}}}
%%%%%%%

\begin{document}

\AddToShipoutPicture{\BackgroundPic}


\renewcommand{\appendixtocname}{Annexes}
\DeclareGraphicsExtensions{.pdf,.png,.jpg}

\begin{titlepage}
\setlength{\parindent}{0cm}

\begin{center}

% Upper part of the page
 \begin{figure}[!h]
\includegraphics[bb=-550 -10 -250 20, scale=0.7]{./logo.pdf}
\end{figure}
% logo.pdf: 612x792 pixel, 72dpi, 21.59x27.94 cm, bb=0 0 612 792


\vspace{4cm}
\rule{\linewidth}{.5pt}
\vspace{2mm}


\begin{center}
{\LARGE GRAND CERCLE MOBILE - GCM}

\vspace{1cm}


{\Huge \bf CAHIER DES CHARGES}


\end{center}


\vspace{1cm}

%===================================================
\begin{center}
$ $\\
\large{ \textbf{Luiza CICONE - Jérémy KREIN - Jérémy LUQUET - Paul MAYER}}\\
\large{ \textbf{ISI - IF}}
$ $\\
\end{center}
\rule{\linewidth}{.5pt}


\vfill

% Bottom of the page

{\large Mai 2012}

\end{center}
\end{titlepage}

\tableofcontents

\newpage


\section{Définition du besoin}

\subsection{Scénario pour mettre en avant le problème}

La vie associative à Grenoble INP est très riche, une multitude d'associations sont présentes pour dynamiser la vie étudiante.

Les événements proposés aux étudiants sont très nombreux, et les différents outils de communication les promouvant ne sont pas adaptés.

Actuellement les principaux outils utilisés pour la communication sont les mails, les affiches et Facebook.

D'une part les étudiants ne consultent pas toujours régulièrement leurs mails et ne prennent pas le temps de les lire attentivement D'autre part, les informations concernant les événements dans ces mails sont souvent noyées au milieu d'autres informations (libération de colocation, offres de stage, etc.).\\

Sur Facebook les informations sont perdues au milieu de tous les commentaires, les événements en tout genre ayant lieu dans toute la France, etc.\\

Les affiches ont aussi leurs limites, nous ne pouvons pas en imprimer beaucoup et donc seulement les étudiants directement visés ont des affiches dans les locaux de leur école.

Par exemple un étudiant Phelma ne pourra pas s'informer par rapport à un événement sportif de l'Ense3, comme il s'agit des écoles dans des bâtiments différents.
Nous présentons ici des scénarios qui empêchent une communication optimale des événements auprès des étudiants.\\

{\bf Scénario 1}

Alexandre est un nouvel étudiant à l'Ense3 et il est très intéressé par les événements sportifs. Il a participé aux événements organisés par le Bureau des Sports de son école, mais il aimerait aussi pouvoir participer aux événements sportifs d'autres écoles qui ne sont pas directement promus dans son école. Comme il est en première année, il passe beaucoup de temps à l'école, il s'occupe des taches administratives (inscription, assurance, etc.) et donc il n'a pas accès facilement à son ordinateur pour chercher les événements qui l'intéressent.

\textit{Il y a des étudiants ont besoin d'être informés facilement sur certains types d'événements}\\

{\bf Scénario 2}

Des amis se retrouvent en ville pour passer la soirée ensemble. Après avoir passé un peu de temps dans un bar ils cherchent quelque chose à faire pour la suite de la soirée. Marion sort son iPhone pour chercher sur Facebook les événements de ce soir afin de repérer les événements du jour et choisir avec ses amis. Elle se perd dans toutes les informations inutiles partagées par ses amis et n'a pas accès a certaines informations étant donné elle n'est pas connectée avec certaines associations.

\textit{Certains étudiants ont besoin d'avoir accès partout aux informations concernant les associations de Grenoble INP}

\subsection{Modèle de l'utilisateur}

Nos utilisateurs seront des étudiants de Grenoble INP. Au sein de ce groupe, nous pouvons différencier plusieurs profils différents.

D'un côté nous avons les étudiants qui s'intéressent beaucoup à la vie associative de Grenoble INP et qui sont engagés dans des associations. Ils sont souvent très intéressés par tous les événements que réalisent les différentes associations de Grenoble INP, que ce soit pour y participer ou seulement

pour en être informé. Ce profil d'utilisateur veut donc pouvoir être au courant de tout ce qui fait l'actualité.

D'un autre côté, nous avons des étudiants qui ont chacun leurs goûts et occupations propres. Ils sont intéressés uniquement par des événements d'un certain type (soirée, événement sportif; etc) ou d'événements d'associations précises (leur BDE, le club Rock]. Ceux-ci veulent être informés des événements de leur choix.

\newpage

\section{Vision de la solution}

Nous avons choisi de développer une application smartphone dédiée à la communication du Grand Cercle et de la vie associative de Grenoble INP. L'objectif principal est de permettre aux étudiants d'être informés tout le temps des événements organisés par ces associations. Parmi les objectifs secondaires se trouve la communication des nouveautés de la vie étudiante à Grenoble et les différents avantages proposés au étudiants de Grenoble INP.

\subsection{Événements}

Les étudiants doivent pouvoir visualiser toutes les informations concernant un événement telles que la date, l'heure, le lieu, le thème, le prix, etc. Aussi ils auront la possibilité d'exporter un événement précis dans leur calendrier. En ce qui concerne la visualisation, l'écran ne doit pas être chargé d'informations inutiles comme une longue description.

Comme cette fonctionnalité est la plus importante, on doit pouvoir y accéder rapidement. Nous prenons donc en compte la possibilité d'avoir comme page d'accueil l'affichage des événements.

La visualisation des événements doit pouvoir se faire de plusieurs manières en fonction du choix de l'utilisateur. Nous avons considéré trois manières différentes d'organiser cette information : l'affichage de type calendrier qui est indispensable, l'affichage de type liste et un affichage avec les quatre prochains événements.

Pour l'affichage calendrier, l'information présentée est minimale, étant donné que l'espace ne permet pas d'intégrer beaucoup d'informations. L'affichage de type liste présente plus d'informations telles le nom, la date et le lieu et éventuellement une image. L'affichage spécial des prochains événements doit avoir un design agréable constitué notamment de l'affiche de l'événement.

\subsection{Préférences}

L'utilisateur doit pouvoir choisir les modalités d'affichage ainsi que l'information qu'il veut voir. Pour cela, l'application faut proposer des filtres par type d'événement et par association. Une autre fonctionnalité nécessaire est la configuration des notifications.

\subsection{Nouveautés}

Les étudiants doivent avoir des informations concernant la vie étudiante à Grenoble. Cette information sera affichée sous la forme d'une liste avec le titre et un sommaire et optionnellement une image.

\subsection{Bons plans}

Les étudiants doivent pouvoir trouver les informations concernant les avantages de la CVA et de la carte étudiante. Pour cela, chaque bon plan doit présenter l'offre, un contact et un plan d'accès.

\subsection{Recherche}

L'application doit fournir un moteur de recherche global pour que les utilisateurs puissent trouver facilement les informations concernant les événements ou bons plans. Cette recherche doit fonctionner par mots clés.

\subsection{Contact}

Les étudiants doivent pouvoir facilement contacter les associations. Il faut donc prévoir un contact sous la forme d'une adresse mail ou un numéro de téléphone pour chaque association.

\newpage

\section{Méthodes}

\subsection{Étude de la concurrence}
La vie associative à Grenoble INP est dense, et toutes les associations - ou presque - possèdent leur propre site internet ou ont un espace dédié sur le site du Grand Cercle.
Cependant, malgré une visibilité importante sur internet, aucune association n'a encore développé d'application pour smartphone. Cette application serait donc une première du point de vue de la communication pour une association à Grenoble INP.

\subsection{Étude qualitative}
Nous avons interviewé plusieurs étudiants de Grenoble INP avec des profils différents afin de mieux cerner les attentes des utilisateurs potentiels de cette application smartphone. Pour ne pas biaiser notre étude, nous avons préparé un questionnaire précis à réponses ouvertes.
Les entretiens ont tous été enregistrés à l'aide d'un dictaphone afin de ne pas oublier une idée importante lors de la restitution des idées. 

\subsection{Étude quantitative}
En considérant les résultats lors de la restitution des idées obtenues grâce à l'étude qualitative, nous avons élaboré un second questionnaire. Ce questionnaire a permis d'avoir des avis plus nombreux sur ces différentes idées, et ainsi d'évaluer les fonctionnalités qui paraissent indispensables.

\newpage

\section{Bilan}

\subsection{Exigences fonctionnelles}


\subsection{Exigences non fonctionnelles}


\appendix
\addappheadtotoc

\newpage

\section{Questionnaire utilisé pour l'étude qualitative}

 \textbf {\large 1. Informations sur l'interview}

	NOM de l'interviewer
	
	Prénom de l'interviewer
	
	Date (JJ/MM/AAAA)
	
	Heures Début - Fin
	
	Lieu

\vspace{.3cm}

 \textbf {\large 2. Informations sur l'interviewé(e)}

	NOM
	
	Prénom
	
	Ecole : 
	 Ense3,
	 Ensimag,
	 Esisar,
	 Génie Industriel,
	 Pagora,
	 Phelma,
	 Autres
	
	Promo :
	 1A,
	 2A,
	 3A,
	 4A ou +,
	 Césure,
	 Autres
	
	Profil : 
	 Grand Cercleux,
	 Cercleux,
	 Associatifs,
	 Elus étudiants,
	 Lambda

\vspace{.3cm}

 \textbf {\large 3. Bilan de l'interview}

Quels événements ont marqué l'entretien ?
\textit{Interruption, téléphone, etc.}

Quelles informations essentielles retenez vous de cette rencontre ?

Notez l'intérêt de l'entretien sur 5p


\vspace{.3cm}

 \textbf {\large 4. Adhérents et Services}

Êtes vous adhérent au Grand Cercle
\textit{On est adhérent au Grand Cercle quand on possède la CVA}

Pourquoi êtes vous / n'êtes vous pas adhérent au Grand Cercle ?

Connaissez vous les avantages dont vous bénéficiez en tant qu'adhérent du Grand Cercle ?
\textit{C'est à dire les avantages de la CVA}

Voyez vous des inconvénients à être adhérent au Grand Cercle ?
\textit{C'est à dire les inconvénients de la CVA}


\vspace{.2cm}
 \textbf {\large 5. Événements Associatifs et Vous}

Participez vous à des événements associatifs ?
\textit{Soirées, Événements sportifs, etc.}

Si oui $\rightarrow$ Vous participez aux événements de quelles associations ?

Si non $\rightarrow$ Pourquoi ne participez vous pas ?

A quels types d'événements vous participez ?


\vspace{.3cm}

 \textbf {\large 6. Événement et Communication}

Comment vous informez vous sur les différents événements ?

Trouvez vous que l'information est facile d'accès ?
\textit{Parler des limites des moyens de s'informer existants}


La communication est - elle bien menée par les associations ?

Êtes vous déjà passé à côté d'un événement auquel vous auriez aimé participer si vous en aviez été informé ?

Que trouvez vous de bien dans les moyens de communication actuels ?


Que trouvez vous de mauvais dans les moyens de communication actuels ?


\vspace{.3cm}

 \textbf {\large 7. Usage et Scénarios}

Que pensez vous de l'idée d'une application mobile dédiée à la communication du Grand Cercle et de la vie associative de Grenoble INP ?


Dans quelles circonstances utiliseriez vous une t-elle application ?
\textit{Un téléphone est transportable partout...}


Pouvez vous imaginer un ou plusieurs scénarios où vous utilisez cette application sur votre mobile ou celui d'un(e) ami(e) ?
\textit{Exemple : nous sommes dans un bar, nous ne savons pas quoi faire, etc.}

Qu'attendez vous d'une telle application ?


\vspace{.3cm}

 \textbf {\large 8. Environnement}

Possédez vous un Smartphone ?

Êtes vous familier avec l'utilisation d'applications sur Smartphone ?

Pour vous quels sont les avantages de ces applications ?

Pour vous quels sont les inconvénients de ces applications ?

Utilisez vous de telles applications pour vous informer ? (restaurant, météo, etc.)
\textit {Ne pas les influencer, sauf vraiment si l'interviewé(e) n'a vraiment rien à dire}


\vspace{.3cm}

 \textbf {\large 9. Application et Fonctionnalités}\\
\textit{Dans cette partie, nous parlons de l'application Grand Cercle}

Quelles informations aimeriez vous recevoir ?


Comment aimeriez vous les recevoir ?
Les informations


Quelles fonctionnalités incontournables voyez vous ?


Quelles fonctionnalités annexes voyez vous ?


Avez vous d'autres idées ?

\newpage

\section{Compte-rendus des entretiens}

 \textbf {\large Entretien no. 1}
Arnaud

\textbf{Date, heure, lieu : }
22/05/2012, 12h40, Ensimag

\textbf{Profil : }
Étudiant cercleux

\textbf{École : }
Ensimag

\textbf{Résumé}
	\begin{itemize}
		\item application rapide et fluide
		\item calendrier avec tous les événements
		\item les jours avec événements se distinguent, code couleur pour le type d'événement
		\item envoi automatique de message/notification comme rappel, information
		\item surtout événementiel
		\item proposition de colocations
		\item il veut tous les type d'événements : sport, festif + etc.
	\end{itemize}

\textbf{Événements marquants}
??

\textbf{Impression}
??

%%%%%%%%%%%%%%%%%%%%%%%%%%%


\vspace{.3cm}

 \textbf {\large Entretien no. 2}
Adrien

\textbf{Date, heure, lieu : }
22/05/2012, 13h15, Ensimag

\textbf{Profil : }
Étudiant impliqué dans l'associatif : pôle Gala

\textbf{École : }
Ensimag

\textbf{Résumé}
	\begin{itemize}
		\item application très ergonomique très intuitive
		\item pas de pub
		\item C’est chiant de lire un mail, trop d'effort
	- Informations brèves
	- Calendrier
	- Le logo du grand cercle
	- Des boutons correspondants au types d’événements
	- Un calendrier propre au type d’événement (soirée, sport, bda)
	- Bref ! informations qu’il faut, pas trop de bouton
	- Notification sur les nouveaux évènements
	- Possibilité de commentaire
	- Possibilité d’abonnement à l’événement pour recevoir les infos au fur et à mesure
	\end{itemize}

\textbf{Événements marquants}
??

\textbf{Impression}
??

%%%%%%%%%%%%%%%%%%%%%%%%%%%


\vspace{.3cm}

 \textbf {\large Entretien no. 3}
Yaya

\textbf{Date, heure, lieu : }
22/05/2012, 20h15, Colocation

\textbf{Profil : }
Étudiant cercleux

\textbf{École : }
Ensimag

\textbf{Résumé}
	\begin{itemize}
		\item ?
		\item ?
	\end{itemize}

\textbf{Événements marquants}
??

\textbf{Impression}
??

%%%%%%%%%%%%%%%%%%%%%%%%%%%


\vspace{.3cm}

 \textbf {\large Entretien no. 4}

\textbf{Date, heure, lieu : }
21/05/2012, 16h32, Local des élus étudiant

\textbf{Profil : }
Élu étudiant

\textbf{École : }
Ense3

\textbf{Résumé}
	\begin{itemize}
		\item barre de recherche nécéssaire en cas de connexion internet de mauvaise qualité
		\item un affichage avec calendrier permet d'avoir une vision globale plus simple
		\item classification des événements
		\item possibiliter d'activer/désactiver les notifications
		\item textes très bref pour pouvoir être lus très rapidement
	\end{itemize}
\vspace{.25cm}

\textbf{Événements marquants}
Aucun


\textbf{Impression} 3,5/5

Très communiquant, mais ne va pas vraiment au fond de ses idées.

%%%%%%%%%%%%%%%%%%%%%%%%%%%


\vspace{.3cm}

 \textbf {\large Entretien no. 5}

\textbf{Date, heure, lieu : }
21/05/2012, 19h31, Colocation

\textbf{Profil : }
Étudiant cercleux et grand cercleux


\textbf{École : }
Phelma

\textbf{Résumé}
	\begin{itemize}
		\item classification des événements par association et par type
		\item signaler quand un album photo sort
		\item faire différents onglets pour la navigation
		\item personnaliser l'interface est un plus
	\end{itemize}
\vspace{.25cm}


\textbf{Événements marquants}

Aucun

\textbf{Impression} 3/5

Un peu timide au début, pas très concerné, puis au cours de l'entretien les idées ont commencer à murir

%%%%%%%%%%%%%%%%%%%%%%%%%%%


\vspace{.3cm}

 \textbf {\large Entretien no. 6}

\textbf{Date, heure, lieu : }
21/05/2012, 19h33, Colocation

\textbf{Profil : }
Étudiant, grand cerlceux et ancien cercleux


\textbf{École : }
Phelma

\textbf{Résumé}
	\begin{itemize}
		\item nécéssité d'avoir des informations centralisées et concises, pas comme sur Facebook
		\item il serait intéressant de pouvoir exporter des événements dans le calendrier du smartphone
		\item avoir des informations plus administratives peut aussi intéresser pas mal de monde
		\item l'application doit pouvoir envoyer des notifications sur le smartphone
		\item il faut pouvoir régler les préférences, filtrer les notifications pour ne pas recevoir trop d'informations qui ne font pas partie de nos centres d'intérêts
	\end{itemize}
\vspace{.25cm}


\textbf{Événements marquants}

Interruption de l'enregistrement sonore suite à un appel reçu par l'interviewé, puis reprise.

\textbf{Impression} 4,5/5

Entretien très constructif, avec beaucoup d'idées très intéressante et réalisable. Interviewé concerné et qui n'hésite pas à donner son avis sans se limiter.


%%%%%%%%%%%%%%%%%%%%%%%%%%%


\vspace{.3cm}

 \textbf {\large Entretien no. 7}

\textbf{Date, heure, lieu : }
22/05/2012, 9h11, MINP
\textbf{Profil : }
Étudiant grand cercleux et cercleux


\textbf{École : }
Ense3

\textbf{Résumé}
	\begin{itemize}
		\item présentation des associations de Grenoble INP
		\item possibilité de choisir les notifications que l'on souhaite recevoir
		\item affichage différent pour les infos qui sont les plus récentes pour les mettre en évidence
		\item création d'un espace dédié aux bons plans et aux annonces de co-voiturage
	\end{itemize}
\vspace{.25cm}

\textbf{Événements marquants}

Aucun

\textbf{Impression} 3/5

Tendance à se limiter dans ses idées, par peur de dire des choses impossible à réaliser. Assez déçu par certaines applications qui envoient des notifications sans arrêt, c'est un point crucial.
%%%%%%%%%%%%%%%%%%%%%%%%%%%


\vspace{.3cm}
 \textbf {\large Entretien no. 8}

\textbf{Date, heure, lieu : }
22/05/2012, 9h47, MINP

\textbf{Profil : }
Étudiant grand cercleux


\textbf{École : }
Ense3

\textbf{Résumé}
	\begin{itemize}
		\item l'application doit être rapide et intuitive
		\item il faut pouvoir accéder aux informations déjà consultées dans le cas ou il ni a pas de connexion internet disponible
		\item ne connaît pas les bons plans de la carte étudiant
	\end{itemize}
\vspace{.25cm}

\textbf{Événements marquants}

Aucun

\textbf{Impression} 2/5

Pas très communicatif

%%%%%%%%%%%%%%%%%%%%%%%%%%%


\vspace{.3cm}

 \textbf {\large Entretien no. 9}

\textbf{Date, heure, lieu : }
21/05/2012, 20h15, Colocation

\textbf{Profil : }
Étudiant grand cercleux

\textbf{École : }
Ense3, année césure

\textbf{Résumé}
	\begin{itemize}
		\item ?
		\item ?
	\end{itemize}

\textbf{Événements marquants}
??

\textbf{Impression}
??

%%%%%%%%%%%%%%%%%%%%%%%%%%%


\vspace{.3cm}

 \textbf {\large Entretien no. 10}
Emile

\textbf{Date, heure, lieu : }
21/05/2012, 20h15, Colocation

\textbf{Profil : }
Étudiant

\textbf{École : }
Ensimag

\textbf{Résumé}
	\begin{itemize}
		\item ?
		\item ?
	\end{itemize}

\textbf{Événements marquants}
??

\textbf{Impression}
??

%%%%%%%%%%%%%%%%%%%%%%%%%%%

\vspace{.3cm}

 \textbf {\large Entretien no. 11}
Simon

\textbf{Date, heure, lieu : }
21/05/2012, 20h15, Colocation

\textbf{Profil : }
Élu étudiant, ancien cercleux

\textbf{École : }
Ensimag

\textbf{Résumé}
	\begin{itemize}
		\item ?
		\item ?
	\end{itemize}

\textbf{Événements marquants}
??

\textbf{Impression}
??

%%%%%%%%%%%%%%%%%%%%%%%%%%%


\vspace{.3cm}

 \textbf {\large Entretien no. 12}
Firmin

\textbf{Date, heure, lieu : }
23/05/2012, 19h00, Colocation

\textbf{Profil : }
Étudiant grand cercleux

\textbf{École : }
Ense3, année césure

\textbf{Résumé}
	\begin{itemize}
		\item ?
		\item ?
	\end{itemize}

\textbf{Événements marquants}
??

\textbf{Impression}
??

%%%%%%%%%%%%%%%%%%%%%%%%%%%

\vspace{.3cm}

 \textbf {\large Entretien no. 13}
Justine

\textbf{Date, heure, lieu : }
23/05/2012, 19h30, Colocation

\textbf{Profil : }
Étudiant grand cercleux, élu étudiant

\textbf{École : }
Pagora

\textbf{Résumé}
	\begin{itemize}
		\item ?
		\item ?
	\end{itemize}

\textbf{Événements marquants}
??

\textbf{Impression}
??

%%%%%%%%%%%%%%%%%%%%%%%%%%%


\vspace{.3cm}

 \textbf {\large Entretien no. 14}
Denis

\textbf{Date, heure, lieu : }
24/05/2012, 20h15, Maison de l'INP

\textbf{Profil : }
Étudiant grand cercleux

\textbf{École : }
Génie Industriel

\textbf{Résumé}
	\begin{itemize}
		\item ?
		\item ?
	\end{itemize}

\textbf{Événements marquants}
??

\textbf{Impression}
??

%%%%%%%%%%%%%%%%%%%%%%%%%%%


\vspace{.3cm}

 \textbf {\large Entretien no. 15}
Simon Papet

\textbf{Date, heure, lieu : }
25/05/2012, 12h00, Kfet Pagora

\textbf{Profil : }
Étudiant 

\textbf{École : }
Pagora

\textbf{Résumé}
	\begin{itemize}
		\item ?
		\item ?
	\end{itemize}

\textbf{Événements marquants}
??

\textbf{Impression}
??
\newpage

\section{Questionnaire utilisé pour l'étude quantitative}


\end{document}

