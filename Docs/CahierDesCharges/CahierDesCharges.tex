\documentclass[a4paper, 11px]{article}

\usepackage[french]{babel}
\usepackage[utf8]{inputenc}
\usepackage{fancyhdr}
\usepackage{lastpage}
\usepackage{graphicx}
\usepackage{rotating}
\usepackage{textcomp}
\usepackage{xspace}
\usepackage[toc,page]{appendix}
\usepackage{array}
\usepackage{amssymb}
\usepackage{enumerate}
\usepackage{enumitem}
\usepackage{eso-pic}

% \usepackage{needspace}


%%%%%%
 
\usepackage{listings}

\lstset{
  morekeywords={},
  sensitive=f,
  morecomment=[l]--,
  morestring=[d]",
  showstringspaces=false,
  basicstyle=\small\ttfamily,
  keywordstyle=\bf\small,
  commentstyle=\itshape,
  stringstyle=\sf,
  extendedchars=true,
  columns=[c]fixed
}

% CI-DESSOUS: conversion des caractères accentués UTF-8 
% en caractères TeX dans les listings...
\lstset{
  literate=%
  {À}{{\`A}}1 {Â}{{\^A}}1 {Ç}{{\c{C}}}1%
  {à}{{\`a}}1 {â}{{\^a}}1 {ç}{{\c{c}}}1%
  {É}{{\'E}}1 {È}{{\`E}}1 {Ê}{{\^E}}1 {Ë}{{\"E}}1% 
  {é}{{\'e}}1 {è}{{\`e}}1 {ê}{{\^e}}1 {ë}{{\"e}}1%
  {Ï}{{\"I}}1 {Î}{{\^I}}1 {Ô}{{\^O}}1%
  {ï}{{\"i}}1 {î}{{\^i}}1 {ô}{{\^o}}1%
  {Ù}{{\`U}}1 {Û}{{\^U}}1 {Ü}{{\"U}}1%
  {ù}{{\`u}}1 {û}{{\^u}}1 {ü}{{\"u}}1%
}

%%%%%%%%%%
% TAILLE DES PAGES (A4 serré)

\setlength{\parindent}{1cm}
\setlength{\parskip}{1ex}
\setlength{\textwidth}{17cm}
\setlength{\textheight}{22,7cm}
\setlength{\oddsidemargin}{-.7cm}
\setlength{\evensidemargin}{-.7cm}


\renewcommand{\labelenumi}{\arabic{enumi}.} 
\renewcommand{\labelenumii}{\arabic{enumi}.\arabic{enumii}}
\renewcommand{\labelenumiii}{\arabic{enumi}.\arabic{enumii}.\arabic{enumiii}}

%%%%%%%%%%

\newcommand\BackgroundPic{
\put(0,0){
\parbox[b][\paperheight]{\paperwidth}{%
\vfill
\includegraphics[width=\paperwidth,height=\paperheight,
keepaspectratio]{background.jpg}%
\vfill
}}}
%%%%%%%

\begin{document}

\AddToShipoutPicture{\BackgroundPic}


\renewcommand{\appendixtocname}{Annexes}
\DeclareGraphicsExtensions{.pdf,.png,.jpg}

\begin{titlepage}
\setlength{\parindent}{0cm}

\begin{center}

% Upper part of the page
 \begin{figure}[!h]
\includegraphics[bb=-550 -10 -250 20, scale=0.7]{./logo.pdf}
\end{figure}
% logo.pdf: 612x792 pixel, 72dpi, 21.59x27.94 cm, bb=0 0 612 792


\vspace{4cm}
\rule{\linewidth}{.5pt}
\vspace{2mm}


\begin{center}
{\LARGE GRAND CERCLE MOBILE - GCM}

\vspace{1cm}


{\Huge \bf CAHIER DES CHARGES}


\end{center}


\vspace{1cm}

%===================================================
\begin{center}
$ $\\
\large{ \textbf{Luiza CICONE - Jérémy KREIN - Jérémy LUQUET - Paul MAYER}}\\
\large{ \textbf{ISI - IF}}
$ $\\
\end{center}
\rule{\linewidth}{.5pt}


\vfill

% Bottom of the page

{\large Mai 2012}

\end{center}
\end{titlepage}

\tableofcontents

\newpage


\section{Définition du besoin}

\subsection{Scénario pour mettre en avant le problème}

La vie associative à Grenoble INP est très riche, une multitude d'associations sont présentes pour dynamiser la vie étudiante.

Les événements proposés aux étudiants sont très nombreux, et les différents outils de communication les promouvant ne sont pas adaptés.

Actuellement les principaux outils utilisés pour la communication sont les mails, les affiches et Facebook.

D'une part les étudiants ne consultent pas toujours régulièrement leurs mails et ne prennent pas le temps de les lire attentivement D'autre part, les informations concernant les événements dans ces mails sont souvent noyées au milieu d'autres informations (libération de colocation, offres de stage, etc.).

Sur Facebook les informations sont perdues au milieu de tous les commentaires, les événements en tout genre ayant lieu dans toute la France, etc.

Les affiches ont aussi leurs limites, nous ne pouvons pas en imprimer beaucoup et donc seulement les étudiants directement visés ont des affiches dans les locaux de leur école.

Par exemple un étudiant Phelma ne pourra pas s'informer par rapport à un événement sportif de l'Ense3, comme il s'agit des écoles dans des bâtiments différents.
Nous présentons ici des scénarios qui empêchent une communication optimale des événements auprès des étudiants.

{\bf Scénario 1}

Alexandre est un nouvel étudiant à l'Ense3 et il est très intéressé par les événements sportifs. Il a participé aux événements organisés par le Bureau des Sports de son école, mais il aimerait aussi pouvoir participer aux événements sportifs d'autres écoles qui ne sont pas directement promus dans son école. Comme il est en première année, il passe beaucoup de temps à l'école, il s'occupe des taches administratives (inscription, assurance, etc.) et donc il n'a pas accès facilement à son ordinateur pour chercher les événements qui l'intéressent.

\textit{Certains étudiants ont besoin d'être informés facilement sur certains types d'événements}\\

{\bf Scénario 2}

Des amis se retrouvent en ville pour passer la soirée ensemble. Après avoir passé un peu de temps dans un bar ils cherchent quelque chose à faire pour la suite de la soirée. Marion sort son iPhone pour chercher sur Facebook les événements de ce soir afin de repérer les événements du jour et choisir avec ses amis. Elle se perd dans toutes les informations inutiles partagées par ses amis et n'a pas accès a certaines informations étant donné elle n'est pas amie (sur Facebook) avec certaines associations.

\textit{Certains étudiants ont besoin d'avoir accès partout aux informations concernant les associations de Grenoble INP}

\subsection{Modèle de l'utilisateur}

Les utilisateurs ciblés seront des étudiants de Grenoble INP. Au sein de ce groupe, nous pouvons différencier plusieurs profils différents.

D'un côté nous avons les étudiants qui s'intéressent beaucoup à la vie associative de Grenoble INP et qui sont engagés dans des associations. Ils sont souvent très intéressés par tous les événements que réalisent les différentes associations de Grenoble INP, que ce soit pour y participer ou seulement pour en être informé. Ce profil d'utilisateur veut donc pouvoir être au courant de tout ce qui fait l'actualité.

D'un autre côté, nous avons des étudiants qui ont chacun leurs goûts et occupations propres. Ils sont intéressés uniquement par des événements d'un certain type (soirée, événement sportif; etc) ou d'événements d'associations précises (leur BDE, le club Rock). Ceux-ci veulent être informés des événements de leur choix.

\newpage

\section{Méthodes}

Nous avons effectué une analyse dans trois parties. 

\subsection{Étude de la concurrence}
La vie associative à Grenoble INP est dense, et presque toutes les associations possèdent leur propre site internet ou ont un espace dédié sur le site du Grand Cercle.
Cependant, malgré une visibilité importante sur internet, aucune association n'a encore développé d'application pour smartphone. Cette application serait donc une première du point de vue de la communication pour une association à Grenoble INP.

Nous avons pris conscience de l'existence de l'application gratuite Bukkett, présente sur les plateformes iOS et Android, permettant aux BDE de faire de la communication sur un campus donné. Sur une vingtaine d'applications Bukkett de BDE, seules quelques unes sont maintenues à jour ce qui laisse penser que les étudiants les utilisent, toutes les autres étant vierges ou fortement incomplètes. 

De plus, Bukket ne nous permet pas de récupérer automatiquement les données du site internet du Grand Cercle, ce qui doublerait le travail des responsables communication qui devraient entrer l'information en double, sur le site et sur Bukkett. Elle propose également beaucoup trop de fonctions ce qui rend la navigation floue et compliquée. Aussi, après avoir parcourus l'application Bukkett, nous avons relevé plusieurs bug  (exportation dans le calendrier, partage, etc.). Le plus important c'est que la vie associative de Grenoble INP est très particulière avec le Grand Cercle et ses nombreux clubs et associations affiliés. Une telle application n'est donc pas adaptée à nos besoins.

\subsection{Étude qualitative}
Pour la démarche qualitative, nous avons interviewé plusieurs étudiants de Grenoble INP avec des profils différents. Afin de mieux cerner les attentes des utilisateurs potentiels de cette application smartphone, nous avons choisi des personnes qui font partie du Grand Cercle (7 étudiants), des Cercles des Élèves, des associations de Grenoble INP, des élus étudiants (6 étudiants),  mais aussi des étudiants qui ne s'impliquent pas dans la vie associative (3 étudiants).

Nous avons conçu un questionnaire (annexe \ref{q_qualitatif})
pour guider les entretiens d'environ 20 minutes.
Les entretiens ont tous été enregistrés à l'aide d'un dictaphone afin de ne pas oublier une idée importante lors de la restitution des idées. 
Après avoir analysé tous les entretiens (annexe \ref{entretiens}), nous avons retenu plusieurs idées et fonctionnalités pour l'application :
\begin {itemize}
	\item calendrier, événements
	\item nouveautés de la vie étudiante
	\item bons plans
	\item petites annonces
	\item localisation des différents lieux
	\item notifications et personnalisation des notifications
	\item achat des places
	\item personnalisation de l'interface
	\item photos
	\item chat
\end{itemize}

\subsection{Étude quantitative}
En considérant les résultats lors de la restitution des idées obtenues grâce à l'étude qualitative, nous avons élaboré un second questionnaire (annexe \ref{q_quantitatif}). Ce questionnaire a permis d'avoir des avis plus nombreux sur ces différentes idées, et ainsi d'évaluer les fonctionnalités qui paraissent indispensables décrites dans la section suivante.
Nous avons obtenu 356 réponses des étudiants de tous les écoles de Grenoble. Les résultats sont présentés dans l'annexe \ref{stats} .


\newpage

\section{Vision de la solution}

Nous avons choisi de développer une application smartphone dédiée à la communication du Grand Cercle et de la vie associative de Grenoble INP. L'objectif principal est de permettre aux étudiants d'être informés tout le temps des événements organisés par ces associations. Parmi les objectifs secondaires se trouve la communication des nouveautés de la vie étudiante à Grenoble et les différents avantages proposés au étudiants de Grenoble INP.

\vspace{.3cm}

 \textbf {\large Événements}

Les étudiants doivent pouvoir visualiser toutes les informations concernant un événement telles que la date, l'heure, le lieu, le thème, le prix, etc. Aussi ils auront la possibilité d'exporter un événement précis dans leur calendrier. Cette fonctionnalité est la plus importante, on doit pouvoir y accéder rapidement. 

La visualisation des événements doit pouvoir se faire de plusieurs manières en fonction du choix de l'utilisateur. Nous avons considéré trois manières différentes d'organiser cette information : l'affichage de type calendrier qui est indispensable, l'affichage de type liste et un affichage avec les quatre prochains événements.

Pour l'affichage calendrier, l'information présentée est minimale, étant donné que l'espace ne permet pas d'intégrer beaucoup d'informations. L'affichage de type liste présente plus d'informations telles le nom, la date et le lieu et éventuellement une image. L'affichage spécial des prochains événements doit avoir un design agréable constitué notamment de l'affiche de l'événement.

\vspace{.3cm}

 \textbf {\large Préférences}

L'utilisateur doit pouvoir choisir les modalités d'affichage ainsi que l'information qu'il veut voir. Pour cela, l'application faut proposer des filtres par type d'événement et par association. Une autre fonctionnalité nécessaire est la configuration des notifications.

\vspace{.3cm}

 \textbf {\large Nouveautés}

Les étudiants doivent avoir des informations concernant la vie étudiante à Grenoble. Cette information sera affichée sous la forme d'une liste avec le titre et un sommaire et optionnellement une image.

\vspace{.3cm}

 \textbf {\large Bons plans}

Les étudiants doivent pouvoir trouver les informations concernant les avantages de la CVA et de la carte étudiante. Pour cela, chaque bon plan doit présenter l'offre, un contact et un plan d'accès.

\vspace{.3cm}

 \textbf {\large Autres informations}

Les étudiants doivent pouvoir facilement contacter les associations. Il faut donc prévoir un contact sous la forme d'une adresse mail ou un numéro de téléphone pour chaque association.

Les informations légales générales doivent aussi être disponibles directement sur l'application.
\vspace{.3cm}

\newpage

\section{Bilan}
Au vue des différentes idées extraites des études qualitatives et quantitatives, un certain nombre d'exigences fonctionnelles et non fonctionnelles ont émergé pour cette application smartphone.
\subsection{Exigences du point de vue de l'utilisateur}
\subsubsection{Exigences fonctionnelles}
\begin{itemize}
	\item[\textbullet] \textbf{Consultation des événements}\\
	L'application doit permettre de consulter les événements publiés sur le site internet \texttt{http://www.grandcercle.org} en présentant les informations sous trois formes différentes :
	\begin{itemize}
		\item une liste triée par date
		\item un calendrier, en différenciant visuellement les jours où un événement à lieu des jours où aucun événement n'a lieu.
		\item un affichage des quatre prochains événements à venir
	\\
	\end{itemize}

	\item[\textbullet] \textbf{Consultation des news}\\
	L'application devra aussi permettre d'afficher les news publiées sur le site internet \texttt{http://www.grandcercle.org}, et de les présenter sous forme de liste.\\
	\item[\textbullet] \textbf{Configuration des préférences}\\
	L'application doit donner la possibilité à l'utilisateur de configurer ses préférences. En effet, l'utilisateur doit pouvoir :
	\begin{itemize}
		\item restreindre l'affichage des événements et des news aux seules associations et type d'événements sélectionnés dans leurs préférences
		\item activer/désactiver les notifications
		\item quand les notifications sont activées, sélectionner les types d'événements et les associations pour lesquels l'utilisateur souhaite recevoir des notifications
	\end{itemize}
\end{itemize}

\subsubsection{Exigences non fonctionnelles}

L'ergonomie d'une application est un réel moteur d'utilisation. En effet, la première chose qui fait qu'une personne utilise une application smartphone est la rapidité avec laquelle elle accède à l'information qu'elle veut. L'application doit donc être à la fois légère du point de vue place mémoire mais également minimiser le nombre de clics et de menus pour trouver les informations. Cela implique donc une certaine simplicité d'utilisation avec peu de fenêtres différentes, une information centralisée afin de rendre la navigation intuitive. 

 Le choix des préférences mentionné dans la partie précédente doit donc permettre à l'utilisateur d'effectuer des réglages qui filtrent les données affichées afin de ne pas surcharger l'écran d'information inintéressantes pour l'utilisateur.

 L'utilisateur sera également sensible à la synchronisation des données de l'application avec celles du site. Les informations présentées devront donc être actualisées régulièrement afin que l'application ait une réelle valeur ajoutée.

\subsection{Exigences techniques}
L'application iOS doit marcher sur les dispositifs iPhone 3GS, 4 et 4S avec les version du système d'exploitation 4.3 et 5.1.

L'application doit être maintenable facilement afin que les prochains étudiants élus au Grand Cercle puissent avoir la possibilité s'ils le souhaitent de modifier l'application suivant leurs besoins, d'en faire des mises à jour, etc...


\appendix
\addappheadtotoc

\newpage

\section{Questionnaire utilisé pour l'étude qualitative}
\label{q_qualitatif}

 \textbf {\large 1. Informations sur l'interview}

	NOM de l'interviewer
	
	Prénom de l'interviewer
	
	Date (JJ/MM/AAAA)
	
	Heures Début - Fin
	
	Lieu

\vspace{.3cm}

 \textbf {\large 2. Informations sur l'interviewé(e)}

	NOM
	
	Prénom
	
	Ecole : 
	 Ense3,
	 Ensimag,
	 Esisar,
	 Génie Industriel,
	 Pagora,
	 Phelma,
	 Autres
	
	Promo :
	 1A,
	 2A,
	 3A,
	 4A ou +,
	 Césure,
	 Autres
	
	Profil : 
	 Grand Cercleux,
	 Cercleux,
	 Associatifs,
	 Elus étudiants,
	 Lambda

\vspace{.3cm}

 \textbf {\large 3. Bilan de l'interview}

Quels événements ont marqué l'entretien ?
\textit{Interruption, téléphone, etc.}

Quelles informations essentielles retenez vous de cette rencontre ?

Notez l'intérêt de l'entretien sur 5p


\vspace{.3cm}

 \textbf {\large 4. Adhérents et Services}

Êtes vous adhérent au Grand Cercle
\textit{On est adhérent au Grand Cercle quand on possède la CVA}

Pourquoi êtes vous / n'êtes vous pas adhérent au Grand Cercle ?

Connaissez vous les avantages dont vous bénéficiez en tant qu'adhérent du Grand Cercle ?
\textit{C'est à dire les avantages de la CVA}

Voyez vous des inconvénients à être adhérent au Grand Cercle ?
\textit{C'est à dire les inconvénients de la CVA}


\vspace{.2cm}
 \textbf {\large 5. Événements Associatifs et Vous}

Participez vous à des événements associatifs ?
\textit{Soirées, Événements sportifs, etc.}

Si oui $\rightarrow$ Vous participez aux événements de quelles associations ?

Si non $\rightarrow$ Pourquoi ne participez vous pas ?

A quels types d'événements vous participez ?


\vspace{.3cm}

 \textbf {\large 6. Événement et Communication}

Comment vous informez vous sur les différents événements ?

Trouvez vous que l'information est facile d'accès ?
\textit{Parler des limites des moyens de s'informer existants}


La communication est - elle bien menée par les associations ?

Êtes vous déjà passé à côté d'un événement auquel vous auriez aimé participer si vous en aviez été informé ?

Que trouvez vous de bien dans les moyens de communication actuels ?


Que trouvez vous de mauvais dans les moyens de communication actuels ?


\vspace{.3cm}

 \textbf {\large 7. Usage et Scénarios}

Que pensez vous de l'idée d'une application mobile dédiée à la communication du Grand Cercle et de la vie associative de Grenoble INP ?


Dans quelles circonstances utiliseriez vous une t-elle application ?
\textit{Un téléphone est transportable partout...}


Pouvez vous imaginer un ou plusieurs scénarios où vous utilisez cette application sur votre mobile ou celui d'un(e) ami(e) ?
\textit{Exemple : nous sommes dans un bar, nous ne savons pas quoi faire, etc.}

Qu'attendez vous d'une telle application ?


\vspace{.3cm}

 \textbf {\large 8. Environnement}

Possédez vous un Smartphone ?

Êtes vous familier avec l'utilisation d'applications sur Smartphone ?

Pour vous quels sont les avantages de ces applications ?

Pour vous quels sont les inconvénients de ces applications ?

Utilisez vous de telles applications pour vous informer ? (restaurant, météo, etc.)
\textit {Ne pas les influencer, sauf vraiment si l'interviewé(e) n'a vraiment rien à dire}


\vspace{.3cm}

 \textbf {\large 9. Application et Fonctionnalités}\\
\textit{Dans cette partie, nous parlons de l'application Grand Cercle}

Quelles informations aimeriez vous recevoir ?


Comment aimeriez vous les recevoir ?
Les informations


Quelles fonctionnalités incontournables voyez vous ?


Quelles fonctionnalités annexes voyez vous ?


Avez vous d'autres idées ?

\newpage

\section{Compte-rendus des entretiens}
\label{entretiens}


 \textbf {\large Entretien no. 1}

\textbf{Date, heure, lieu : }
22/05/2012, 12h40, Ensimag

\textbf{Profil : }
Membre du cercle des élèves Ensimag

\textbf{École : }
Ensimag

\textbf{Résumé}

	\begin{itemize}
		\item application rapide et fluide
		\item calendrier des événements
		\item code couleur pour les types d'événements
		\item utilisation de notifications ou messages pour informer
		\item proposition de colocations
	\end{itemize}

\vspace{.25cm}
\textbf{Événements marquants}	
	\begin{itemize}
		\item portable qui vibre
	\end{itemize}

\vspace{.25cm}
\textbf{Impression} 2.5/5 

Quelques bonnes idées

%%%%%%%%%%%%%%%%%%%%%%%%%%%


\vspace{.3cm}

 \textbf {\large Entretien no. 2}

\textbf{Date, heure, lieu : }
22/05/2012, 13h15, Ensimag

\textbf{Profil : }
Étudiant

\textbf{École : }
Ensimag

\textbf{Résumé}
	\begin{itemize}
		\item application ergonomique et intuitive
		\item information brève
		\item calendrier des événements, un calendrier par type d'événement
		\item notification pour les nouveaux événements
		\item possibilité de s'abonner à un événement pour recevoir en notification les informations à son sujet
		\item pas de publicité
	\end{itemize}

\vspace{.25cm}
\textbf{Événements marquants}
	\begin{itemize}
		\item portable qui vibre
		\item entrée de quelqu'un dans la salle de l'interview
	\end{itemize}

\vspace{.25cm}
\textbf{Impression} 3.5/5 

Très intéressé, dynamique

%%%%%%%%%%%%%%%%%%%%%%%%%%%


\vspace{.3cm}

 \textbf {\large Entretien no. 3}

\textbf{Date, heure, lieu : }
22/05/2012, 20h15, Colocation

\textbf{Profil : }
Étudiant cercleux

\textbf{École : }
Ensimag

\textbf{Résumé}
	\begin{itemize}
		\item application très rapide, information disponible immédiatement
		\item tous les événements de la vie associative de Grenoble INP, et de tout Grenoble ce serait mieux
		\item calendrier des événements
		\item événements et descriptions pour répondre aux questions éventuelles
		\item utilisation des notifications
	\end{itemize}

\vspace{.25cm}
\textbf{Événements marquants}
	\begin{itemize}
		\item aucun
	\end{itemize}

\vspace{.25cm}
\textbf{Impression} 2.5/5 

Beaucoup d'idées mélangées, manque de clarté

%%%%%%%%%%%%%%%%%%%%%%%%%%%


\vspace{.3cm}

 \textbf {\large Entretien no. 4}

\textbf{Date, heure, lieu : }
21/05/2012, 16h32, Local des élus étudiant

\textbf{Profil : }
Élu étudiant

\textbf{École : }
Ense3

\textbf{Résumé}
	\begin{itemize}
		\item barre de recherche nécessaire en cas de connexion internet de mauvaise qualité
		\item un affichage avec calendrier permet d'avoir une vision globale plus simple
		\item classification des événements
		\item possibilité d'activer/désactiver les notifications
		\item textes très bref pour pouvoir être lus très rapidement
	\end{itemize}
\vspace{.25cm}

\textbf{Événements marquants}
Aucun


\textbf{Impression} 3,5/5

Très communiquant, mais ne va pas vraiment au fond de ses idées.

%%%%%%%%%%%%%%%%%%%%%%%%%%%


\vspace{.3cm}

 \textbf {\large Entretien no. 5}

\textbf{Date, heure, lieu : }
21/05/2012, 19h31, Colocation

\textbf{Profil : }
Étudiant cercleux


\textbf{École : }
Phelma

\textbf{Résumé}
	\begin{itemize}
		\item classification des événements par association et par type
		\item signaler quand un album photo sort
		\item faire différents onglets pour la navigation
		\item personnaliser l'interface est un plus
	\end{itemize}
\vspace{.25cm}


\textbf{Événements marquants}

Aucun

\textbf{Impression} 3/5

Un peu timide au début, pas très concerné, puis au cours de l'entretien les idées ont commencer à mûrir

%%%%%%%%%%%%%%%%%%%%%%%%%%%


\vspace{.3cm}

 \textbf {\large Entretien no. 6}

\textbf{Date, heure, lieu : }
21/05/2012, 19h33, Colocation

\textbf{Profil : }
Étudiant grand cerlceux


\textbf{École : }
Phelma

\textbf{Résumé}
	\begin{itemize}
		\item nécéssité d'avoir des informations centralisées et concises, pas comme sur Facebook
		\item il serait intéressant de pouvoir exporter des événements dans le calendrier du smartphone
		\item avoir des informations plus administratives peut aussi intéresser pas mal de monde
		\item l'application doit pouvoir envoyer des notifications sur le smartphone
		\item il faut pouvoir régler les préférences, filtrer les notifications pour ne pas recevoir trop d'informations qui ne font pas partie de nos centres d'intérêts
	\end{itemize}
\vspace{.25cm}


\textbf{Événements marquants}

Interruption de l'enregistrement sonore suite à un appel reçu par l'interviewé, puis reprise.

\textbf{Impression} 4,5/5

Entretien très constructif, avec beaucoup d'idées très intéressante et réalisable. Interviewé concerné et qui n'hésite pas à donner son avis sans se limiter.


%%%%%%%%%%%%%%%%%%%%%%%%%%%


\vspace{.3cm}

 \textbf {\large Entretien no. 7}

\textbf{Date, heure, lieu : }
22/05/2012, 9h11, Maison de l'INP

\textbf{Profil : }
Étudiant cercleux


\textbf{École : }
Ense3

\textbf{Résumé}
	\begin{itemize}
		\item présentation des associations de Grenoble INP
		\item possibilité de choisir les notifications que l'on souhaite recevoir
		\item affichage différent pour les infos qui sont les plus récentes pour les mettre en évidence
		\item création d'un espace dédié aux bons plans et aux annonces de co-voiturage
	\end{itemize}
\vspace{.25cm}

\textbf{Événements marquants}

Aucun

\textbf{Impression} 3/5

Tendance à se limiter dans ses idées, par peur de dire des choses impossible à réaliser. Assez déçu par certaines applications qui envoient des notifications sans arrêt, c'est un point crucial.
%%%%%%%%%%%%%%%%%%%%%%%%%%%


\vspace{.3cm}
 \textbf {\large Entretien no. 8}

\textbf{Date, heure, lieu : }
22/05/2012, 9h47, Maison de l'INP

\textbf{Profil : }
Étudiant grand cercleux


\textbf{École : }
Ense3

\textbf{Résumé}
	\begin{itemize}
		\item l'application doit être rapide et intuitive
		\item il faut pouvoir accéder aux informations déjà consultées dans le cas ou il ni a pas de connexion internet disponible
		\item ne connaît pas les bons plans de la carte étudiant
	\end{itemize}
\vspace{.25cm}

\textbf{Événements marquants}

Aucun

\textbf{Impression} 2/5

Pas très communicatif

%%%%%%%%%%%%%%%%%%%%%%%%%%%


\vspace{.3cm}

 \textbf {\large Entretien no. 9}

\textbf{Date, heure, lieu : }
21/05/2012, 14h00, Colocation

\textbf{Profil : }
Étudiant grand cercleux

\textbf{École : }
Ense3, année césure

\textbf{Résumé}
	\begin{itemize}
		\item il faut centraliser toutes les informations via le site du Grand Cercle
		\item application qui doit être très simple d'utilisation
		\item agenda avec événements et pas seulement de Grenoble INP (manifestations organisées par la ville notamment)
		\item agenda par date et par association (au choix)
		\item s’inscrire d'abord à l’événement pour recevoir des notifications
		\item lien vers l’itinéraire mappy ou google maps, ça serait cool
		\item recevoir les infos sur les événements auquel on participe (mise à jour, nouvelle
     info), choisir de recevoir ou pas.
		\item pourquoi pas des photos (retour sur évènements)
		\item bons plans (avec la cva notamment)

	\end{itemize}

\textbf{Événements marquants}
reception de messages sur iPhone de la personne interrogée

\textbf{Impression}
4/5

Assez sceptique au début de l'entretien. S'est pris au jeu et a donné un avis assez différent, de par sa participation à des événements hors Grenoble INP

%%%%%%%%%%%%%%%%%%%%%%%%%%%


\vspace{.3cm}

 \textbf {\large Entretien no. 10}

\textbf{Date, heure, lieu : }
22/05/2012, 12h15, Ensimag

\textbf{Profil : }
Étudiant

\textbf{École : }
Ensimag

\textbf{Résumé}
	\begin{itemize}
		\item ne participe pas aux événements associatifs, il est intéressé par les événements sportifs, mais pas par les soirées
		\item n'est pas très bien informé et pense que la communication est assez statique : mail, affiche et sur Facebook trop d'informations
		\item possède un smartphone BlackBerry, mais trouve que les applications pour cela ne sont pas bien faites (pas rapides)
		\item utilise des applications d'actualités, des flux RSS, le Twitter mais pas trop le Facebook
		\item calendrier avec tous les événements avec des notifications à choix (mail, sms)
		\item acheter sa place sur internet pour les événements
		\item informations offres, vie étudiante Grenoble
	\end{itemize}

\vspace{.25cm}
\textbf{Événements marquants}
	\begin{itemize}
		\item interruption appel téléphone
	\end{itemize}
\vspace{.25cm}

\textbf{Impression}
3/5

Réponses assez brèves

%%%%%%%%%%%%%%%%%%%%%%%%%%%

\vspace{.3cm}

 \textbf {\large Entretien no. 11}

\textbf{Date, heure, lieu : }
21/05/2012, 13h00, Ensimag

\textbf{Profil : }
Élu étudiant

\textbf{École : }
Ensimag

\textbf{Résumé}
	\begin{itemize}

		\item pense que c'est facile à s'informer et qu'il y a parfois trop d'information
		\item participe aux événements des cercles et de quelques associations et s'informe sur Facebook et parfois des affiches.
		\item pense que la communication se fait assez tard et est incomplète
		\item pas de smartphone, il utilise les réseaux sociaux et des applications pour s'informer et communiquer
		\item pense qu'une application est bonne pour les gens qui suivent pas Facebook, offre un moyen en plus pour communiquer
		\item informations sur les annonces, bons plans
		\item des notifications, mais pas de mails
	\end{itemize}

\vspace{.25cm}
\textbf{Événements marquants}
	\begin{itemize}
		\item interruption téléphone
	\end{itemize}

\vspace{.25cm}
\textbf{Impression}
2/5

Investi beaucoup de temps à s'informer. Pas beaucoup de nouvelles idées.

%%%%%%%%%%%%%%%%%%%%%%%%%%%


\vspace{.3cm}

 \textbf {\large Entretien no. 12}

\textbf{Date, heure, lieu : }
23/05/2012, 19h00, Colocation

\textbf{Profil : }
Étudiant grand cercleux

\textbf{École : }
Génie Industriel

\textbf{Résumé}
	\begin{itemize}
		\item trouver surtout les événements, des informations courtes
		\item recherche de soirée, ou recherche un événement en question
		\item les applications sont condensés, plus rapides
		\item recherche par date, par association, type, le calendrier
		\item détails de prix, date, lieu, plus d'information sur le site
		\item des notifications avec filtre
		\item informations de bases sur Grenoble INP
	\end{itemize}

\textbf{Événements marquants}
Aucun

\textbf{Impression}
3/5

Assez interessé

%%%%%%%%%%%%%%%%%%%%%%%%%%%

\vspace{.3cm}

 \textbf {\large Entretien no. 13}


\textbf{Date, heure, lieu : }
23/05/2012, 19h30, Colocation

\textbf{Profil : }
Étudiant grand cercleux, élu étudiant

\textbf{École : }
Pagora

\textbf{Résumé}
	\begin{itemize}
		\item pense que l'application peut être très utile
		\item fait un tri, sélectionne les informations qui intéresse
		\item utilise avec les amis comme ne possède pas un smartphone
		\item information répertoriée, classée
		\item 2 affichages : chercher quoi faire ce soir, cherche un événement d'un cercle spécifique
		\item information synthétique
		\item recevoir des rappels, mais pas forcement
		\item informations sur les bons plans
	\end{itemize}

\textbf{Événements marquants}
Aucun

\textbf{Impression}
3,5/5

Assez dynamique et interessée


%%%%%%%%%%%%%%%%%%%%%%%%%%%


\vspace{.3cm}

 \textbf {\large Entretien no. 14}

\textbf{Date, heure, lieu : }
24/05/2012, 20h15, Maison de l'INP

\textbf{Profil : }
Étudiant grand cercleux

\textbf{École : }
Génie Industriel

\textbf{Résumé}
	\begin{itemize}
		\item participe à beaucoup d'événements, s'informe par Facebook, bouche à oreille
		\item application beaucoup plus lisible que Facebook et rapide (trouver information tout de suite)
		\item les bons plans proposés par le Grand Cercle
		\item l'application doit pas être très lourde, information rapide, pas tout le site dans l'application
		\item contacts du Grand Cercle, des cercles
		\item choisir des options par association, aussi type de notification, informations des mises à jour
	\end{itemize}

\textbf{Événements marquants}
	\begin{itemize}
		\item beaucoup d'interruptions
	\end{itemize}

\textbf{Impression}
3,5/5

Beaucoup de bonnes idées, mais un approche asses subjective en tant que responsable partenaires.

%%%%%%%%%%%%%%%%%%%%%%%%%%%


\vspace{.3cm}

 \textbf {\large Entretien no. 15}

\textbf{Date, heure, lieu : }
25/05/2012, 12h30, Kfet Pagora

\textbf{Profil : }
Étudiant 

\textbf{École : }
Pagora

\textbf{Résumé}
	\begin{itemize}
		\item participe aux événements de l'école, la communication est bien menée au sein de l'école 
		\item s'informe sur Facebook principalement, mail, affiches
		\item une application permette de tout rassembler et donne plus envie de chercher des infos, pas rater des soirées
		\item chercher quoi faire ce soir
		\item les informations de l'événement, avec un plan 
		\item recevoir des notification, mais pas forcement, pas de mails
		\item choisir les préférences par association
		\item informations sur Grenoble, transport, navigation
		\item contact des personnes qui organisent
	\end{itemize}

\textbf{Événements marquants}
Aucun

\textbf{Impression}
3,5/5

Souligné bien les idées, mais pas trop de nouveautés.

%%%%%%%%%%%%%%%%%%%%%%%%%%%

\vspace{.3cm}

 \textbf {\large Entretien no. 16}

\textbf{Date, heure, lieu : }
22/05/2012, 20h15, Maison de Grenoble INP

\textbf{Profil : }
Membre du Grand Cercle

\textbf{École : }
Ensimag

\textbf{Résumé}
	\begin{itemize}
		\item application simple, rapide avec prise en main immédiate
		\item application belle graphiquement
		\item application légère, son téléchargement doit être rapide
		\item application customisable, un thème par école de Grenoble INP par exemple
		\item information disponible immédiatement
		\item calendrier des événements pour les dates
		\item liste des événements pour avoir plus d'informations
		\item consultation des news au sujet des associations
		\item consultation des news au sujet de Grenoble INP
		\item filtre pour que l'on puisse consulter que ce qui nous intéresse
		\item pas de notification ou alors possibilité de changer les paramètres pour ne pas les recevoir
		\item pas de pub
	\end{itemize}

\vspace{.25cm}
\textbf{Événements marquants}
	\begin{itemize}
		\item entrée de personnes dans la salle de l'entretien
	\end{itemize}

\vspace{.25cm}
\textbf{Impression} 4.5/5 

Beaucoup d'idées précises, très clair, très intéressé

\newpage

\section{Questionnaire utilisé pour l'étude quantitative}
\label{q_quantitatif}

 \textbf {\large Nous sommes 4 étudiants de l'Ensimag, membres du Grand Cercle. Dans le cadre de notre projet de spécialité, nous concevons une application iOS et Android dédiée à la communication du Grand Cercle. Ce questionnaire dure environ 5 minutes, aidez nous à concevoir l'application que vous souhaiteriez utiliser !}\\


\textbf{1. Dans quelle école de Grenoble INP étudiez vous ? }

\indent Ense3\\
\indent Ensimag\\
\indent Esisar\\
\indent Génie Industriel\\
\indent Pagora\\
\indent Phelma\\
\indent CPP\\
\indent Je ne suis pas étudiant à Grenoble INP


\textbf{2. Quelle est votre promo ? }

\indent 1A\\
\indent 2A\\
\indent 3A\\
\indent 4A +\\
\indent Année de césure\\
\indent Je ne suis pas étudiant à Grenoble INP\\


\textbf{3. Cochez les différentes associations dont vous faites partie : }

\indent Grand Cercle\\
\indent Cercles des élèves\\
\indent BDS\\
\indent BDA\\
\indent Elus étudiants\\
\indent Clubs et associations de Grenoble INP\\
\indent Autres\\
\indent Aucune\\


\textbf{4. Connaissez vous le Grand Cercle ? }

    Oui
    Non


\textbf{5.Possédez-vous la CVA }

    Oui
    Non


\textbf{6. Auriez vous aimé avoir plus de détails concernant les différents avantages proposés par la CVA ?}

    Oui
    Non


\textbf{7. Aux événements de quelle(s) association(s) participez-vous ? }

\indent le Cercle de ton école\\
\indent les Cercles des autres écoles\\
\indent le Grand Cercle\\
\indent le Bureau International\\
\indent clubs et assos affiliés au Grand Cercle\\
\indent Autre : 


\textbf{8. Quels types d'événements vous interessent ? }

\indent soirées\\
\indent événements sportifs\\
\indent événements culturels\\
\indent événements des clubs et assos\
\indent Autre : 


\textbf{9. Comment faites vous pour trouver des informations sur les différents événements ? }

\indent Site internet du Grand Cercle\\
\indent Sites internet des Cercles\\
\indent Sites internet des clubs et assos\\
\indent Facebook\\
\indent Mail général\\
\indent Affiches\\
\indent Bouche à oreille\\
\indent Autre : 


\textbf{10. Les informations sont-elles faciles d'accès ? }

    Oui
    Non


\textbf{11. L'information sur ces événements vous semble-t-elle suffisante ? } Donnez une note

	1 	2 	3 	4 	5 	

1 = Pas du tout ; 5 = Completement

\textbf{12. Vous est-il déjà arrivé de ne pas avoir été informé de la tenue d'un événement qui aurait pu vous intéresser ? }

    Oui
    Non


\textbf{13. Souhaiteriez-vous avoir plus de retours sur les événements proposés ? (photos, chiffres liés à l'événement, résultats dans le cas d'un événement sportif, etc...)}

    Oui
    Non


\textbf{14. Pensez-vous que la gestion des objets trouvés lors d'un événement peut être améliorée ? }

    Oui
    Non


\textbf{15. Que pensez vous de l'idée d'une application mobile dédiée à la communication du Grand Cercle et de la vie associative de Grenoble INP ?}

	1 	2 	3 	4 	5 	

1 = Inutile ; 5 = Indispensable

\textbf{16. Sur quoi portent les informations que vous souhaiteriez trouver dans une telle application ? }

\indent Le Grand Cercle\\
\indent Les Cercles\\
\indent Les club et associations affiliés au Grand Cercle\\
\indent Les élus étudiants\\
\indent Les partenaires du Grand Cercle


\textbf{17. Parmi les fonctionnalités suivantes, lesquelles vous paraissent indispensables/inutiles ?  Classez chacune des fonctionnalités suivant les critères suivants : Inutile, Peu utile, Utile, Souhaitable, Indispensable }\\
\indent Notifications concernant les événements \\							
\indent Photos 							\\
\indent Tchat 							\\
\indent Calendrier des événements exportable dans le calendrier du smartphone 	\\	
\indent Bons plans 		\\					
\indent Information sur les événements 	\\						
\indent Contacts utiles 				\\			
\indent Actualités / News sur les événements\\ 							
\indent Code couleur suivant le type des événements\\ 							
\indent Ajouter des commentaires sur des événements / photos\\ 					
\indent Actualités / News sur la vie étudiante 					\\		
\indent Personnalisation du thème 							\\

\textbf{18. Pensez-vous à d'autres fonctionnalités importantes qui n'ont pas été mentionnées précédemment ? Donnez quelques précisions}

\textbf{19. Possédez vous un Smartphone ? }

    Oui
    Non


\textbf{20. Si oui, quel type de smartphone ?}

\indent iOS (iPhone)\\
\indent Android\\
\indent Blackberry\\
\indent Autre : 


\textbf{21. Êtes vous familier avec l'utilisation d'applications sur Smartphone ? }

    Oui
    Non


\textbf{22. Si oui, quels critères retenez-vous pour évaluer la qualité d'une application smartphone ? Classez suivant : Nuisible, Préjudiciable, Indifférent, Important, Fondamental}\\
\indent Rapidité 		\\					
\indent Simplicité 	\\						
\indent Exhaustivité des informations disponibles 			\\				
\indent Facilité de navigation 							\\
\indent Ergonomie 							\\
\indent Aspect visuel général 					\\		

\textbf{23. Si une application gratuite Grand Cercle était disponible, la téléchargeriez vous ? }

    Oui
    Non


\textbf{24. Souhaitez-vous ajouter quelque chose ?}
\newpage

\section{Résultats de l'étude quantitative}
\label{stats}
\end{document}

